\documentclass[a4paper]{article}
% \usepackage[margin=1.25in]{geometry}
\usepackage[inner=2.0cm,outer=2.0cm,top=2.5cm,bottom=2.5cm]{geometry}
% \usepackage{ctex}
\usepackage{color}
\usepackage{graphicx}
\usepackage{amssymb}
\usepackage{amsmath}
\usepackage{amsthm}
\usepackage{bm}
\usepackage{hyperref}
\usepackage{multirow}
\usepackage{enumerate}

\newcommand{\homework}[5]{
    \pagestyle{myheadings}
    \thispagestyle{plain}
    \newpage
    \setcounter{page}{1}
    \noindent
    \begin{center}
    \framebox{
        \vbox{\vspace{2mm}
        \hbox to 6.28in { {\bf Optimization Methods \hfill #2} }
        \vspace{6mm}
        \hbox to 6.28in { {\Large \hfill #1 \hfill} }
        \vspace{6mm}
        \hbox to 6.28in { {\it Instructor: {\rm #3} \hfill Name: {\rm #4}, StudentId: {\rm #5}}}
        \vspace{2mm}}
    }
    \end{center}
    % \markboth{#4 -- #1}{#4 -- #1}
    \vspace*{4mm}
}

\newenvironment{solution}
{\color{blue} \paragraph{Solution.}}
{\newline \qed}

\begin{document}
%==========================Put your name and id here==========================
\homework{Homework 1}{Fall 2019}{Lijun Zhang}{Chenxiao Gao}{181220014}

\paragraph{Notice}
\begin{itemize}
    \item The submission email is: \textbf{njuoptfall2019@163.com}.
    \item Please use the provided \LaTeX{} file as a template. If you are not familiar with \LaTeX{}, you can also use Word to generate a \textbf{PDF} file.
\end{itemize}



\paragraph{Problem 1: Norms}
~

A function $f:\mathbb{R}^n \rightarrow \mathbb{R}$ with $\text{dom}f=\mathbb{R}^n$ is called a \textit{norm} if
\begin{itemize}
    \item $f$ is nonnegative: $f(x) \geq 0$ for all $x \in \mathbb{R}^n$
    \item $f$ is definite: $f(x) = 0$ only if $x=0$
    \item $f$ is homogeneous: $f(tx)=|t|f(x)$, for all $x \in \mathbb{R}^n$ and $t \in \mathbb{R}$
    \item $f$ satisfies the triangle inequality: $f(x+y) \leq f(x) + f(y)$, for all $x,y \in  \mathbb{R}^n$
\end{itemize}
We use the notation $f(x) = \|x\|$. Let $\| \cdot \|$ be a norm on $\mathbb{R}^{n}$.
The associated dual norm, denoted $\| \cdot \|_*$, is defined as 
$$\|z\|_* = \sup \{z^Tx | \|x\| \leq 1\}$$

\noindent
\begin{enumerate}[a)]
    \item Prove that $\|\cdot\|_*$ is a valid norm.
    \item Prove that the dual of the Euclidean norm ($\ell_2$-norm) is the Euclidean norm, \emph{i.e.}, prove that
$$\|z\|_{2*} = \sup \{z^Tx | \|x\|_2 \leq 1\} = \|z\|_2$$.\\ (\textit{Hint:} Use Cauchy–Schwarz inequality.)
\end{enumerate}
\begin{solution}
    \begin{enumerate}[a)]
        \item We can prove that $f(x)$ satisfies the abovementioned properties:
            \begin{enumerate}[1)]
                \item randomly choose an vector x that satisfies $\|x\|\leq 1$, and it is trivial that $\|-x\|\leq 1$. If $z^Tx\geq 0$, then we have $\|z\|_*=sup\{z^Tx|||x||\leq 1\}\geq 0$, or else $\|z\|_*\geq z^T(-x)\geq0$. Thus, $\|\cdot\|_*$is nonnegative.
                \item based on 1), if $\|z\|_*=0$ then we have $z^Tx,z^T(-x)=-z^Tx\leq 0$, which denotes that for all x $\in \mathbb{R}^n,z^Tx=-z^Tx=0$, and $z^T=0$. if $z^T=0$, it is trivial that $\|z\|_*=0$. Thus, $\|\cdot\|_*$ is definite.
                \item for all $x\in \mathbb{R}^n$ and for all $t \in \mathbb{R}$, 
                
                if $t\geq 0$, then $\|tz\|_*=sup\{tz^Tx|\|x\|\leq 1\}=tsup\{z^Tx|\|x\|\leq 1\}=|t|\|z\|_*$(t is a constant), which means $\|\cdot\|_*$ is homogeneous.

                else if $t\leq 0$, then
                
                \begin{aligned}
                \|tz\|_* &=\|-t(-z)\|_*\\
                &=sup\{(-t)(-z)^T(-x)|\|-x\|\leq 1\}\\
                &=(-t)sup\{(-z)^T(-x)|\|-x\|\leq 1\}\\
                &=|t|sup\{z^Tx|\|-x\|\leq 1\}\\
                &=|t|sup\{z^Tx|\|x\|\leq 1\}\\
                &=|t|\|z\|_*\\
                \end{aligned}

                Therefore, $\|\cdot\|_*$ is homogeneous.
                \item For all $x,y\in v$, $\|x+y\|_*=sup\{x^Tz+y^Tz|\|z\|\leq 1\}\leq sup\{x^Tz|\|z\|\leq 1\}+sup\{y^Tz|\|z\|\leq 1\}=\|x\|_*+\|y\|_*$, and the equality holds if and only if $x^Tz$ and $y^Tz$ achieves their maximum at the same point. Thus $\|\cdot\|$satisfies the triangle inequality.
            \end{enumerate}
        \item From Cauchy–Schwarz inequality, we can show that $|z^Tx|$ achieves maximum when $x=\frac{z}{\|z\|_2}$. Noticing that $\frac{z}{\|z\|_2}\leq 1$, we can conclude that$\|z\|_*=sup\{z^Tx|\|x\|\leq 1\}=\frac{z^Tz}{\|x\|_2}=\|z\|_2$, and the equality holds when $x=\frac{z}{\|z\|_2} or \frac{-z}{\|z\|_2}$, depending on the sign of $z^Tx$. Q.E.D.
    \end{enumerate}
\end{solution}
\paragraph{Problem 2: Affine and Convex Sets}
~

Affine sets $C_{a}$ and convex $C_{c}$ sets are the sets satisfying the constraints below:
\begin{equation}
\begin{aligned}
    \theta x_1 + (1-\theta)x_2 \in C_{a}\\
    \text{s.t. } x_1, x_2 \in C_{a}\\
\end{aligned}
\end{equation}
\begin{equation}
\begin{aligned}
    \theta x_1 + (1-\theta)x_2 \in C_{c}\\
    \text{s.t. } x_1, x_2 \in C_{c}, 0 \geq \theta \leq 1\\
\end{aligned}
\end{equation}

\noindent
\begin{enumerate}[a)]
    % \item Can we hold that $D$ is convex if $D$ is affine?
    \item Is the set $\{  \alpha \in \mathbb{R}^k | p(0)=1, |p(t)|\leq1 \text{ for } \alpha \leq t \leq \beta\}$, where $p(t) = \alpha_1+\alpha_2t+\cdots+\alpha_kt^{k-1}$, affine?
    \item Determine if each set below is convex.
        \begin{enumerate}[1)]
            \item $\{  (x,y)\in \mathbf{R}^2_{++} | x/y \leq 1 \}$.
            \item $\{  (x,y)\in \mathbf{R}^2_{++} | x/y \geq 1 \}$.
            \item $\{  (x,y)\in \mathbf{R}^2_{+} | xy \leq 1 \}$.
            \item $\{  (x,y)\in \mathbf{R}^2_{+} | xy \geq 1 \}$.
            \item $\{  (x,y)\in \mathbf{R}^2 | y = \text{tanh}(x) = \frac{e^{x}-e^{-x}}{e^{x}+e^{-x}}\}$.
        \end{enumerate}
\end{enumerate}

\begin{solution}
    \begin{enumerate}[a)]
        \item Not necessarily.
        
        proof:

        Define $S=\{  \alpha \in \mathbb{R}^k | p(0)=1, |p(t)|\leq1 \text{ for } \alpha \leq t \leq \beta\}$for all $\alpha_1$ and $\alpha_2 \in \mathbb{R}^n$

        Consider $\alpha_1=(1, 0, -1), \alpha_2=(1, 0, -2), t\in [-1, 1]$, we can see that both $\alpha_1$ and $\alpha_2$ belongs to $S$.

        However, for $\alpha_*=\theta \alpha_1+(1-\theta)\alpha_2$, the absolute value of the polynomial $p(t)=1+(\theta-2)t^2$ is not necessarily below 1.

        Thus, the set S is not necessarily affine.
        
        \item Define $S$ as the following sets.
        \begin{enumerate}[1)]
            \item Convex. 

            proof: 
            
            For all $(x_1, y_1) and (x_2,y_2)\in \mathbb{R}_{++}^2$, we have $x_1\leq y_1, x_2\leq y_2$, so $\theta x_1+(1-\theta) x_2\leq \theta y_1+(1-\theta)y_2$. 

            $\because \theta, (1-\theta), y_1, y_2\geq 0$

            $\therefore \frac{\theta x_1+(1-\theta)x_2}{\theta y_1+(1-\theta)y_2}\leq 1$

            Therefore the set S is convex.

            \item Convex.
            
            proof:

            For all $(x_1, y_1) and (x_2,y_2)\in \mathbb{R}_{++}^2$, we have $x_1\geq y_1, x_2\geq y_2$, so $\theta x_1+(1-\theta) x_2\geq \theta y_1+(1-\theta)y_2$. 

            $\because \theta, (1-\theta), y_1, y_2\geq 0$

            $\therefore \frac{\theta x_1+(1-\theta)x_2}{\theta y_1+(1-\theta)y_2}\geq 1$

            Fherefore the set S is convex.

            \item Not convex.
            
            proof:

            Assign $(x_1, y_1)=(2, 1/2), (x_2, y_2)=(2, 1/2), \theta=1/2$

            $\therefore (\theta x_1 + (1-\theta)x_2)(\theta y_1 + (1-\theta)y_2)=(5/4)^2=25/16>1$

            Thus, the set S is not convex.

            \item Convex.
            
            proof:

            $\because x_1y_1\geq 1, x_2y_2\geq 1$

            $\therefore x_1y_2+x_2y_1\geq \frac{x_1}{x_2}+\frac{x_2}{x_1}\geq 2$

            $\therefore (\theta x_1 + (1-\theta)x_2)(\theta y_1 + (1-\theta)y_2)=(x_2y_2)+(\theta-\theta^2)(2-x_1y_2-x_2y_1)$

            $\because (\theta-\theta^2)\leq 0, 2-x_1y_2-x_2y_1\leq 0$

            $\therefore (\theta x_1 + (1-\theta)x_2)(\theta y_1 + (1-\theta)y_2)\geq 1$

            Therefore the set S is convex.

            \item Not Convex.
            
            proof:

            Assign $(x_1, y_1)=(0,0), (x_2, y_2)=(1, \frac{e-e^{-1}}{e+e^{-1}}), \theta=1/2$

            Towever, $\frac 12y_1+\frac 12y_2 \not = tanh(\frac 12x_1+\frac12 x_2)$

            Therefore, the set S is not convex.

        \end{enumerate}
    \end{enumerate}
\end{solution}

\paragraph{Problem 3: Examples}
~
\begin{enumerate}[a)]
\item Let $C \subseteq \mathbb{R}^n$ be the solution set of a quadratic inequality,
    \begin{equation}
        C = \{ x\in \mathbb{R}^n | x^\top Ax+b^\top x+c \leq 0\}\,,
    \end{equation}
    with $A \in \mathbb{S}^n, b \in \mathbb{R}^n, \text{and } c \in \mathbb{R}$.

    \noindent
    \begin{enumerate}[1)]
        \item Show that $C$ is convex if $A \succeq 0$.
        \item Is the following statement true? The intersection of $C$ and the hyperplane defined by $g^\top x+h=0$ is convex if $A+\lambda gg^\top \succeq 0$ for some $\lambda \in \mathbb{R}$.
    \end{enumerate}
\item The polar of $C\subseteq \mathbb{R}^n$ is defined as the set
$$C^\circ = \{ y\in \mathbb{R}^n | y^\top x\leq 1 \text{ for all }x \in C\}$$\,
    \noindent
    \begin{enumerate}[1)]
        \item Show that $C^\circ$ is convex.
        \item What is a polar of a polyhedra?
        \item What is the polar of the unit ball for a norm $||\cdot||$?
        \item Show that if $C$ is closed and convex, with $0 \in C$, then $(C^\circ)^\circ = C$
    \end{enumerate}
\end{enumerate}


\begin{solution}
    \begin{enumerate}[a)]
        \item 
            \begin{enumerate}[1)]
                \item proof:
                
                C is convex

                $\Leftrightarrow $for any $x_1\in C$, $x_2\in C$, and $\theta \in[0, 1]$, $\theta x_1+(1-\theta)x_2\in C$

                $\Leftrightarrow (\theta x_1+(1-\theta)x_2)^TA(\theta x_1+(1-\theta)x_2)+b^T(\theta x_1+(1-\theta)x_2)+c\leq 0$

                $\Leftrightarrow \theta^2x_1^TAx_1+(1-\theta)^2x_2^TAx_2+\theta(1-\theta)x_1^TAx_2+\theta(1-\theta)x_2^TAx_1+b^T(\theta x_1+(1-\theta)x_2)+c\leq 0$

                (considering the shapes of $x_1^TAx_2$ and $x_2^TAx_1$ are both 1*1, $x_1^TAx_2$ actually equals to $x_2^TAx_1$)

                $\Leftrightarrow \theta^2x_1^TAx_1+(1-\theta)^2x_2^TAx_2+2\theta(1-\theta)x_1^TAx_2+b^T(\theta x_1+(1-\theta)x_2)+c\leq 0$

                ~\\

                $\because b^Tx_1\leq -c-x_1^TAx_1, b^Tx_1\leq -c-x_2^TAx_2$
                    
                $\therefore P = \theta^2x_1^TAx_1+(1-\theta)^2x_2^TAx_2+2\theta(1-\theta)x_1^TAx_2+b^T(\theta x_1+(1-\theta)x_2)+c$

                $\leq \theta^2x_1^TAx_1+(1-\theta)^2x_2^TAx_2+2\theta(1-\theta)x_1^TAx_2+\theta(-c-x_1^TAx_1)+(1-\theta)(-c-x_2^TAx_2)+c$

                $= \theta(\theta-1)x_1^TAx_1+\theta(\theta-1)x_2^TAx_2+2\theta(1-\theta)x_1^TAx_2$

                $= \theta(1-\theta)(2x_1^TAx_2-x_1^TAx_1-x_2^TAx_2)$

                $= \theta(1-\theta)(x_1^TA(x_2-x_1)+(x_1-x_2)^TAx_2)$

                $= \theta(1-\theta)(-(x_1-x_2)^TA(x_1-x_2))$

                $\because A\in \mathbb{S}_+^n$

                $\therefore \theta(1-\theta)(-(x_1-x_2)^TA(x_1-x_2))\leq 0$

                $\therefore P \leq 0$

                This means the original proposition 'C is convex' is true.

                Q.E.D.

                \item the statement is true. 
                
                proof:

                Let $D$ stands for the hyperplane defined by $g^Tx+h=0$

                First, it is trivial that any convex combination of $x_1, x_2\in C\cup D$ also belongs to $D$, for hyperplanes are always convex.

                So we only need to prove that the convex combination belongs to $C$.
                
                For any $x_1\in C\cup D $and $x_2\in C\cup D$, we have $g^Tx_1+h=g^Tx_2+h=0$

                $\because g^Tx_1+h=g^Tx_2+h=0$

                $\therefore x_1^Tg=x_2^Tg=0$

                $\therefore x_1^Tgg^Tx_1=x_2^Tgg^Tx_2=0$

                $\therefore (x_1-x_2)^Tgg^T(x_1-x_2)=0$

                $\because A+\lambda gg^T\in \mathbb{S}_+^n$

                $\therefore (x_1-x_2)^T(A+gg^T)(x_1-x_2)\geq 0$

                $\because (x_1-x_2)^Tgg^T(x_1-x_2)=0$

                $\therefore (x_1-x_2)^TA(x_1-x_2)=(x_1-x_2)^T(A+gg^T)(x_1-x_2)-(x_1-x_2)^Tgg^T(x_1-x_2)\geq 0$

                This means the polynomial $P$ defined in 1) satisfies $P\leq 0$

                Therefore, the set $C$ is convex, and $\theta x_1+(1-\theta)x_2\in C \cup D$

                Q.E.D.
                
            \end{enumerate}

        \item 
            \begin{enumerate}[1)]
                \item proof:
                
                For any $y_1, y_2\in C^\circ$, we have $y_1^Tx\leq 1$(for all $x\in C$) and $y_2^Tx\leq 1$(for all $x\in C$).

                $\therefore $ For all $x \in C$and $\theta \in[0, 1]$, $(\theta y_1+(1-\theta)y_2)^Tx=\theta y_1^Tx+(1-\theta)y_2^Tx\leq \theta+(1-\theta)=1$

                $\therefore \theta y_1+(1-\theta)y_2\in C^\circ$

                Therefore $C^\circ$ is convex.

                \item solution:
                
                Suppose $P$ is a bounded polyhedron. (I am not sure whether polyhedra are all bounded, but here I suppose they are)
                
                Therefore, $P$ can be represented by $conv\{v_1, v_2, ...v_k\}$, for k finite, and any element from $P$ can be represented by $\theta_1v_1+...+\theta_kv_k$, where $\theta_i\geq 0$ and $1^T\theta=1$
                
                For $x\in P^\circ$, the inner product of $x$ and $\theta_1v_1+...+\theta_kv_k$ is below 1 no matter how $\theta$ varies.

                So we let $x$ satisfies $x^Tv_i\leq 1, i=1,2...k$, so that $x^T(\theta_1v_1+...+\theta_kv_k)=\theta_1x^Tv_1+...+\theta_kx^Tv_k\leq \theta_1+\theta_2+...+\theta_k=1$

                In the meantime, the set $\{x\in \mathbb{R}^n|x^Tv_i\leq 1, i=1,2...k\}$ is actually a polygedron.

                So, the polar of a polygedron is a polyhedron.

                \item solution:
                
                According to the definition of dual norm $\|\cdot\|_*$: $\|z\|_* = sup\{z^Tx|\|x\|\leq 1\}$,

                the polar of the unit ball for $\|\cdot\|$ is $\{y|\|y\|_*\leq 1\}$

                \item proof:
                
                We prove $(C^\circ)^\circ=C$ by 2 steps.

                $(\supseteq)$: 

                Suppose $x\in C$.

                $\therefore $ for all $y\in C^\circ$, $y^Tx\leq 1$

                $\because y^Tx=(y^Tx)^T=x^Ty$

                $\therefore $ for all $y\in C^\circ$, $x^Ty\leq 1$

                $\therefore x\in (C^\circ)^\circ$

                $\therefore (C^\circ)^\circ \supseteq C$.

                $(\subseteq)$:

                Suppose there exists at least one element $x \in (C^\circ)^\circ$ but $x\not \in C$.

                According to the Separating Hyperplane Theorem, there exists a seperating hyperplane $a^Tx=b$ between the non-intersecting sets $C$ and $(C^\circ)^\circ-C$.

                By adjusting the value of $a$, we can scale down the value of $b$ to 1, so we simply assume the hyperplane here is $a^Tx=1$

                $\because 0\in C$ and $C$ is a closed set

                $\therefore$ for all $z\in C, a^Tz\leq 1$ ; for all $w \in (C^\circ)^\circ, a^Tw > 1$ 

                $\therefore a\in C^\circ$

                However, the supposition we made before indicates that there exists one element $x\in (C^\circ)^\circ-C$, and for x we have $a^Tx > 1$, and this contradicts with $x\in (C^\circ)^\circ$

                Therefore $(C^\circ)^\circ-C=\emptyset$.

                So $(C^\circ)^\circ\subseteq C$

                In conclusion, $(C^\circ)^\circ=C$.


            \end{enumerate}
    \end{enumerate}
\end{solution}

\paragraph{Problem 4: Operations That Preserve Convexity}
~

Suppose $\phi : \mathbb{R}^n \rightarrow \mathbb{R}^m \text{ and } \psi : \mathbb{R}^m \rightarrow \mathbb{R}^p$ are the linear-fractional functions
\begin{equation}
\phi(x) = \frac{Ax+b}{c^\top x + d}, \psi(y) = \frac{Ey+f}{g^\top y + h},
\end{equation}
with domains \textbf{dom }$\phi = \{ x | c^\top x + d > 0 \}$, \textbf{dom }$\psi = \{ y | g^\top y + h > 0 \}$. We associate with $\phi$ and $\psi$ the matrices
\begin{equation}
  \left[\begin{matrix}
   A & b \\
   c^\top & d
  \end{matrix}\right]\,,
  \left[\begin{matrix}
   E & f \\
   g^\top & h
  \end{matrix}\right]\,,
\end{equation}
respectively.

Now, consider the composition $\Gamma$ of $\phi$ and $\psi$, \emph{i.e.}, $\Gamma(x)=\psi(\phi(x))$, with domain
\begin{equation}
\textbf{dom} \Gamma = \{ x \in \textbf{dom } \phi | \phi(x) \in \textbf{dom } \psi\} \,.
\end{equation}

Show that $\Gamma$ is linear-fractional, and that the matrix associate with it is the product
\begin{equation}
  \left[\begin{matrix}
   E & f \\
   g^\top & h
  \end{matrix}\right]
  \left[\begin{matrix}
   A & b \\
   c^\top & d
  \end{matrix}\right]\,.
\end{equation}


\begin{solution}
    \begin{equation}
    \begin{split}
        \because \Gamma(x)&=\psi(\phi(x))\\
        &=\frac{E\frac{Ax+b}{c^Tx+d}+f}{g^T\frac{Ax+b}{c^Tx+d}+h}\\
        &=\frac{(EA+fc^T)x+Eb+fd}{(g^TA+hc^T)x+g^Tb+hd}
    \end{split}
    \end{equation}
    So apparently $\Gamma$ is linear-fractional and it can be associated with the matrix
    \begin{equation}
        \left[\begin{matrix}
         EA+fc^T & Eb+fd \\
         g^TA & hc^T
        \end{matrix}\right],
    \end{equation}
    which is exactly the product
    \begin{equation}
        \left[\begin{matrix}
         E & f \\
         g^\top & h
        \end{matrix}\right]
        \left[\begin{matrix}
         A & b \\
         c^\top & d
        \end{matrix}\right]\,.
    \end{equation}
    Q.E.D.

\end{solution}

\paragraph{Problem 5: Generalized Inequalities}
~

Let $K^*$ be the dual cone of a convex cone $K$. Prove the following
\begin{enumerate}[1)]
    \item $K^*$ is indeed a convex cone.
    \item $K_1 \subseteq K_2$ implies $K_2^* \subseteq K_1^*$.
\end{enumerate}

\begin{solution}
    \begin{enumerate}[1)]
        \item proof:
        
        $\because $ for any $y_1, y_2\in K^\star $, we have $x^Ty_1\geq 0,x^Ty_2\geq 0$ (for any $x\in K$).

        $\therefore $ for any $x \in K$ and $\theta_1,\theta_2 \in [0,+\infty]$, $x^T(\theta_1 y_1+\theta_2y_2)=\theta_1 x^Ty_1+\theta_2x^Ty_2\geq0$.

        $\therefore \theta_1x_1+\theta_2x_2\in K^\star$

        Therefore, $K^\star$ is indeed a cone.

        Q.E.D.

        \item proof:
        
        For any $y\in K_2^\star$, we always have $x^Ty\geq0$ where $x$ is arbitrarily chosen from $K_2$.

        Considering $K_1\subseteq K_2$, $x^Ty\geq 0$ still holds when $x$ is from $K_1$.

        So for any $y\in K_2^\star$, we always have $x^Ty\geq0$ where $x$ is arbitrarily chosen from $K_2$, which means $y\in K_1^\star$

        Therefore, $K_2^\star \subseteq K_1^\star$

        Q.E.D.


    \end{enumerate}
\end{solution}



\end{document}