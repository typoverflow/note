\documentclass[a4paper]{article}
% \usepackage[margin=1.25in]{geometry}
\usepackage[inner=2.0cm,outer=2.0cm,top=2.5cm,bottom=2.5cm]{geometry}
% \usepackage{ctex}
\usepackage{color}
\usepackage{graphicx}
\usepackage{amssymb}
\usepackage{amsmath}
\usepackage{amsthm}
\usepackage{bm}
\usepackage{hyperref}
\usepackage{multirow}
\usepackage{mathtools}
\usepackage{enumerate}
\usepackage[UTF8]{ctex}

\usepackage{algorithm}
\usepackage{algorithmic}
\usepackage{xcolor}
\renewcommand{\algorithmicrequire}{\textbf{Input:}} 
\renewcommand{\algorithmicensure}{\textbf{Output:}}

\newcommand{\homework}[5]{
    \pagestyle{myheadings}
    \thispagestyle{plain}
    \newpage
    \setcounter{page}{1}
    \noindent
    \begin{center}
    \framebox{
        \vbox{\vspace{2mm}
        \hbox to 6.28in { {\bf Optimization Methods \hfill #2} }
        \vspace{6mm}
        \hbox to 6.28in { {\Large \hfill #1 \hfill} }
        \vspace{6mm}
        \hbox to 6.28in { {\it Instructor: {\rm #3} \hfill Name: {\rm #4}, StudentId: {\rm #5}}}
        \vspace{2mm}}
    }
    \end{center}
    % \markboth{#4 -- #1}{#4 -- #1}
    \vspace*{4mm}
}


\newenvironment{solution}
{\color{blue} \paragraph{Solution.}}
{\newline \qed}

\begin{document}
%==========================Put your name and id here==========================
\homework{Homework 3}{Fall 2019}{Lijun Zhang}{高辰潇}{181220014}

\paragraph{Notice}
\begin{itemize}
    \item The submission email is: \textbf{njuoptfall2019@163.com}.
    \item Please use the provided \LaTeX{} file as a template. If you are not familiar with \LaTeX{}, you can also use Word to generate a \textbf{PDF} file.
\end{itemize}
\paragraph{Problem 1: Equality Constrained Least-squares}
~\\
Consider the equality constrained least-squares problem
\begin{gather*}
\begin{matrix}
\text{minimize~~} & \frac{1}{2}\|Ax-b\|_2^2\\
\text{subject to} & Gx=h~~\\
\end{matrix}
\end{gather*}
where $A\in\mathbf{R}^{m\times n}$ with $\mathbf{rank}~A=n$, and $G\in\mathbf{R}^{p\times n}$ with $\mathbf{rank}~G=p$.
\begin{enumerate}[a)]
    \item Derive the Lagrange dual problem with Lagrange multiplier vector $v$.
    \item Derive expressions for the primal solution $x^\star$ and the dual solution $v^\star$.
\end{enumerate}
\begin{solution}
    \begin{enumerate}[a)]
        \item 解:
        
                该问题的Lagrange函数为$L(x, v)=\frac 12(Ax-b)^T(Ax-b)+v^T(Gx-h)=\frac 12 x^TA^TAx+(v^TG-b^TA)x+\frac 12b^Tb-v^Th$,$\nabla_xL(x, v)=A^TAx+G^Tv-A^Tb$.

                得到拉格朗日函数的极小值点为$x=(A^TA)^{-1}(A^Tb-G^Tv)$.

                因此$g(v) = \inf_{x\in D}L(x, v) = \frac 12(v^TG-b^TA)(A^TA)^{-1}(A^Tb-G^Tv)+\frac 12b^Tb-v^Th$.

                所以对偶问题为
                \begin{equation}
                    \begin{split}
                        &\text{maximize}\ \ \frac 12(v^TG-b^TA)(A^TA)^{-1}(A^Tb-G^Tv)+\frac 12b^Tb-v^Th\\
                    \end{split}
                \end{equation}

        \item 解:由于对偶问题是关于变量$v$的无约束优化问题,故$g(v)$对$v$求导得
        
                $\nabla_vg(v)=G(A^TA)^{-1}A^Tb-h-(G(A^TA)^{-1}G^T)v$

                因此,$v^{\star}=(G(A^TA)^{-1}G^T)^{-1}(G(A^TA)^{-1}A^Tb-h)$.

                又由Slater条件,原问题满足强对偶性。进一步由KKT条件,有$A^TAx^\star+G^Tv^\star-A^Tb=0$.

                代入$v^\star$得到$x^\star=(A^TA)^{-1}[A^Tb-G^T(G(A^TA)^{-1}G^T)^{-1}(G(A^TA)^{-1}A^b-h)]$
    \end{enumerate}
\end{solution}

\paragraph{Problem 2: Support Vector Machines}
~\\
Consider the following optimization problem
\begin{gather*}
\begin{matrix}
\text{minimize~~} & \sum_{i=1}^n\max\left(0,1-y_i(w^Tx_i+b)\right)+\frac{\lambda}{2}\|w\|_2^2\\
\end{matrix}
\end{gather*}
where $x_i\in\mathbf{R}^{d},y_i\in \mathbf{R},i=1,\cdots,n$ are given, and $w\in \mathbf{R}^d,b\in\mathbf{R}$ are the variables.
\begin{enumerate}[a)]
    \item Derive an equivalent problem by introducing new variables $u_i,i=1,\cdots,n$ and equality constraints \[u_i=y_i(w^Tx_i+b),i=1,\cdots,n.\]
    \item Derive the Lagrange dual problem of the above equivalent problem.
    \item Give the Karush-Kuhn-Tucker conditions.
\end{enumerate}

\noindent\emph{Hint: Let $\ell(x)=\max(0,1-x)$. Its conjugate function $\ell^\ast(y)=\sup\limits_{x}(yx-\ell(x))=\left\{
\begin{aligned}
&y, \quad-1\leq y\leq0 \\
&\infty,\quad\text{otherwise}
\end{aligned}
\right.$}
\begin{solution}
    \begin{enumerate}[a)]
        \item 解:等价问题为
            \begin{equation*}
                \begin{split}
                    &\text{minimize~~}\ \ \sum_{i=1}^n\max(0, 1-u_i)+\frac{\lambda}2\|w\|_2^2\\
                    &\text{subject}\ \text{to}\ u_i=y_i(w^Tx_i+b), i=1, ...,n.
                \end{split}
            \end{equation*}

        \item 解:拉格朗日函数为$L(v, u_i, w)=\sum_{i=1}^n\max(0, 1-u_i)+\frac \lambda 2\|w\|_2^2 + \sum_{i=1}^n v_i[u_i-y_i(w^Tx_i + b)]$.
        
                $g(v)=\inf_{u}\{\sum_{i=1}^n\max(0, 1-u_i)+\sum_{i=1}^n v_iu_i\}+\inf_w\{\frac \lambda 2 \|w\|_2^2-\sum_{i=1}^nv_iy_iw^Tx_i\}+\inf_b\{-v_iy_ib\}$

                $\nabla_w L=\lambda w-\sum_{i=1}^n(v_iy_ix_i)$,故极值点为$w=\frac 1\lambda\sum_{i=1}^nv_iy_ix_i$

                由Hint知$\inf_{u_i}\{\max(0, 1-u_i)+y_iu_i\}=-\sup_{u_i}\{-v_iu_i-\max(0, 1-u_i)\}=\left\{
                \begin{aligned}
                    &v_i, \quad 0\leq v_i\leq 1 \\
                    &-\infty,\quad\text{otherwise}
                \end{aligned}\right.$

                而$\inf_b\{-v_iy_ib\}=\left\{
                \begin{aligned}
                    &0, \quad v_iy_i=0\\
                    &-\infty \quad otherwise
                \end{aligned}\right.$

                因此$g(v)=\sum_{i=1}^n v_i+\frac 1{2\lambda}(\sum_{i=1}^nv_iy_ix_i)^T(\sum_{i=1}^nv_iy_ix_i)-\sum_{i=1}^n \frac 1\lambda v_i^2y_i^2x_i^Tx_i$,其中$v_i\in[0, 1],v_iy_i=0, i\in\{1, ...,n\}$

                故对偶问题为:
                \begin{equation}
                    \begin{split}
                        &\text{maximize} \ \ g(v)=\sum_{i=1}^n v_i+\frac 1{2\lambda}(\sum_{i=1}^nv_iy_ix_i)^T(\sum_{i=1}^nv_iy_ix_i)-\sum_{i=1}^n \frac 1\lambda v_i^2y_i^2x_i^Tx_i\\
                        &\text{subject}\ \text{to}\quad 0\leq v_i\leq 1\\
                        &\quad \quad\quad\quad \quad\quad y_iv_i=0\\
                    \end{split}
                \end{equation}

        \item 解:KKT条件为:
            \begin{enumerate}[1)]
                \item $u^\star_i=y_i((w^\star)^Tx_i+b)$, $i\in\{1, ...,n\}$
                \item 
            \end{enumerate}
    \end{enumerate}
\end{solution}

\paragraph{Problem 3: Euclidean Projection onto the Simplex}
~\\
Consider the following optimization problem
\begin{gather*}
\begin{matrix}
\text{minimize~~} & \frac{1}{2}\left\|y-x\right\|^2_2\quad\\
\text{subject to} & \mathbf{1}^Ty=r\quad\quad\\
&y\succeq0\quad\quad\quad
\end{matrix}
\end{gather*}
where $r>0$, $x\in \mathbb{R}^n$ is given, and $y\in \mathbf{R}^n$ is the variable. Give an algorithm to solve this problem and prove the correctness of your algorithm.

\noindent\emph{Hint: Derive the Lagrangian of this problem and apply the Karush-Kuhn-Tucker conditions. If you need more hints, please read the following paper \cite{Wang2013}}


\begin{algorithm}
    \caption{Solution}
    \begin{algorithmic}[1]
        \REQUIRE $r>0,x\in \mathbb{R}^n=(x_1, x_2, ...,x_n)$
        \STATE Sort $x$ into $u$: $u_1\geq u_2\geq...\geq u_n$
        \STATE $j=1$
        \WHILE{$u_j+\frac 1j(1-\sum_{i=1}^ju_i)>0$}
            \STATE $j=j+1$
        \ENDWHILE
        \STATE $\rho = j-1$
        \STATE Define $v=-\frac 1\rho (1-\sum_{i=1}^\rho u_i)$
        \ENSURE optimal $y^\star \text{s.t.} y_i=\max\{x_i-v ,0\}, i=1, ...,n$.
    \end{algorithmic}
\end{algorithm}

\begin{solution}
    解:首先给出算法。(见Solution)

    下面证明算法正确性。

    该问题的拉格朗日函数为$L(y, \lambda, v)=\frac 12\|y-x\|_2^2-\lambda^T y+v(1^Ty-r)$.

    故对偶函数$g(\lambda, v)=\inf_{n\in D}L(y, \lambda, v)=-\frac 12(x^T+\lambda^T-v\cdot \mathbf{1}^T)(x+\lambda-v\cdot \mathbf{1})$

    故原问题的对偶问题为
        \begin{equation}
            \begin{split}
                &\text{maximize}\ \ g(\lambda, v)\\
                &\text{subject}\ \text{to}\ \lambda\succeq 0\\
            \end{split}
        \end{equation}

    由于原问题为凸问题,且满足Slater条件,因此原问题的最优解$y^\star$和对偶问题的最优解$(\lambda^\star, v^\star)$应满足KKT条件:

    $y^\star\succeq 0$

    $1^Ty^\star=r$

    $\lambda^\star\succeq 0$

    $\lambda^\star y^\star=0$

    $y^\star-x-\lambda^\star+v^\star \mathbf{1}=0$

    考虑最优解$y^\star$的各分量$y_i$,若$y_i=0$,则由条件4知$\lambda_i \geq0$,故$v^\star -x_i\geq 0$;若$y_i >0$,则$\lambda_i=0$,$v^\star -x_i =-y_i<0$.

    不失一般性,将给定的向量$x$按照从大到小的顺序重新排列各分量,设排列后得到新向量$u=(u_1, u_2, ...,u_n)$。
    
    按照原本各维度的对应关系相应地排序最优解,设最优解为$y^\star={y_1, ...,y_n}$。由上述分析知,在$\lambda_i\geq 0$的情况下数值较小的$u_i$一定对应着较小的$y_i$. 因此有$y_1\geq y_2\geq ...\geq y_{\rho} > y_{\rho+1}=...=y_n=0$.

    又由于$1^Ty^\star=y_1+y_2+...+y_{\rho}$,且对于$y_1, ...,y_m$有$y_i=x_i-v^\star$,因此$\sum_{i=1}^\rho(u_i-v^\star)=1$, $v^\star=-\frac 1\rho (1-\sum_{i=1}^\rho u_i)$

    为了确定$v^\star$,还需要确定$\rho$的大小。对于$j\in\{1, 2, ...,n\}$分三种情况考虑:

    a) 若$j<\rho$,则

    \begin{equation}
        \begin{split}
            &\frac 1j(1-\sum_{i=1}^ju_i)+u_j\\
            &=\frac 1j(1-\sum_{i=1}^\rho u_i+\sum_{j+1}^\rho u_i+ju_j)\\
            &=\frac 1j(-\rho v^\star+\sum_{i=j+1}^\rho u_i+ju_j)\\
            &=\frac 1j(j(u_j-v^\star)+\sum_{i=j+1}^\rho(u_i-v^\star))
        \end{split}
    \end{equation}
        
        由于对于$i\in \{1, 2, ...\rho\}$和$j$都有$u_i-v^\star=y_i>0$,故$\frac 1j(1-\sum_{i=1}^ju_i)+u_j>0$.

    b) 若$j>\rho$,则

    \begin{equation}
        \begin{split}
            &\frac 1j(1-\sum_{i=1}^ju_i)+u_j\\
            &=\frac 1j(1-\sum_{i=1}^\rho u_i-\sum_{\rho+1}^j u_i+ju_j)\\
            &=\frac 1j(-\rho v^\star-\sum_{i=\rho+1}^j u_i+ju_j)\\
            &=\frac 1j(j(u_j-v^\star)-\sum_{i=\rho+1}^j(u_i-v^\star))
        \end{split}
    \end{equation}
    
    由于对于$i\in \{\rho+1, ...,n\}$和$j>\rho$都有$u_i-v^\star=<0$,故$\frac 1j(1-\sum_{i=1}^ju_i)+u_j<0$.

    c) 若$j=\rho$,则有
        
    $\frac 1\rho(1-\sum_{i=1}^\rho u_i)+u_{\rho}=\frac 1\rho(\rho(u_\rho-v))>0$.

    因此,算法第3-6行通过判断$\frac 1j(1-\sum_{i=1}^ju_i)+u_j$的正负性,找到使该式为正值的最大的下标,该下标即为$\rho$. 再由$v^\star=-\frac 1\rho (1-\sum_{i=1}^\rho u_i)$可进一步计算出$v^\star$(算法的第7行)。

    而对于原问题最优解$y^\star$,由KKT条件知它的非零分量$y_i(i\leq\rho)$均满足$y_i=x_i-\rho$,且对于它的为零的分量$y_i(i>\rho)$有$x_i-v^\star\leq 0$. 因此,$y^\star$的各分量$y_i$可统一表示为$y_i=\max\{x_i-v^\star, 0\}$.

    故算法的输出即为原问题的最优解。正确性得证。

\end{solution}

\paragraph{Problem 4: Optimality Conditions}
~\\
Consider the problem
\begin{equation*}
    \begin{split}
        &\text{minimize~~} \quad  \text{tr}(2X) - \log{\det{(3X)}} \\
        &\text{subject to} \quad~  2Xs=y
    \end{split}
\end{equation*}
with variable $X \in \mathbf{S}^n$ and domain $\mathbf{S}_{++}^n$. Here, $y \in \mathbf{R}^n$ and $s \in \mathbf{R}^n$ are given, with $s^T y =1$.
\begin{enumerate}[a)]
    \item Give the Lagrange and then derive the Karush-Kuhn-Tucker conditions.
    \item Verify that the optimal solution is given by
    \begin{equation*}
        X^\star = \frac{1}{2} \left(I + yy^T - \frac{ss^T}{s^T s}\right) .
    \end{equation*}
\end{enumerate}

\begin{solution}
    \begin{enumerate}[a)]
        \item 解:Lagrange函数为$L(X, v)=\text{tr}(2X)-\log \det(3X)+v^T(2Xs-y)$,$\nabla_X L=2I-X^{-1}+vs^T+sv^T$.

                KKT条件为:
                \begin{enumerate}[1)]
                    \item $2Xs=y$
                    \item $2I-X^{-1}+vs^T+sv^T=0$
                \end{enumerate}
                其中$X\succ 0$
        \item 解:
                
                由$2Xs=y$知$s=\frac 12X^{-1}y$

                代入$X^{-1}=2I+vs^T+sv^T$得到$s=\frac 12(2I+vs^T+sv^T)y=y+\frac 12(v+(v^Ty)s)$.

                $s^Ty=y^Ty+\frac 12 v^Ty+\frac 12v^Ty=1$,所以有$y^Ty+v^Ty=1$.

                代入$s=y+\frac 12(v+(v^Ty)s)$可得到$v=(1+y^Ty)s-2y$.

                下面证明,题中所给出的$X^\star$满足KKT条件。
                
                \begin{enumerate}[1)]
                    \item $2Xs=2\cdot \frac 12(I+yy^T-\frac{ss^T}{s^Ts})s=s+y-s=y$.
                    \item 等价于证明$X^{-1}X=(2I+v^Ts+s^Tv)X=I$
                            \begin{equation}
                                \begin{split}
                                    X^{-1}X&=(2I+(v^\star)^Ts+s^Tv)X=I\\
                                    &=(2I+2(1+y^Ty)ss^T-2ys^T-2sy^T)(\frac 12(I+yy^T-\frac{ss^T}{s^Ts}))\\
                                    &=I+(1+y^Ty)ss^T-ys^T-sy^T+yy^T+(1+yy^T)sy^T-yy^T-sy^Tyy^T-\frac{ss^T}{s^Ts}-(1+y^Ty)ss^T+ys^T+\frac{ss^T}{s^Ts}\\
                                \end{split}
                            \end{equation}
                
                因此题目所给定的$X$满足KKT条件。又由于$X\succ 0$,故$X$是原问题的最优解。





\end{solution}

\begin{thebibliography}{1}

\bibitem{Wang2013}gugu
Weiran Wang, and Miguel \'{A}. Carreira-Peroi\~{n}\'{a}n.
\newblock Projection onto the probability simplex: An efficient algorithm with a simple proof, and an application.
\newblock \emph{arXiv:1309.1541}, 2013.

\end{thebibliography}

\end{document}
